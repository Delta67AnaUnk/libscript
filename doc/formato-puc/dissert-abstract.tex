Applications written in two programming languages, in order to optimize
parts where performance is critical or to obtain extensibility through
user-written scripts, are commonplace nowadays. There are several
ways to obtain this kind of interoperability; ideally, however, a
language should provide a foreign language interface (FLI), allowing
the programmer to send and receive both data and function calls to
the external language. 

This work discusses the main issues involving the design of APIs for
the integration of language environments within C~applications. We
present the main problems faced in the interaction between code executed
in an environment with inherently dynamic characteristics such as
a scripting language and C~code. We compare the approaches employed
by five languages when handling communication between the data spaces
of C and the embedded runtime environment and the consequences of
these approaches in memory management, as well as sharing of code
between the C~application and that from the scripting language.

We illustrate the differences of the APIs of those languages and their
impact in the resulting code of a C~application through a case study.
Different scripting languages were embedded as plugins for a library,
which on its turn exposes to client applications a generic scripting
API. This way, the code of each plugin allows us to observe in a clear
and isolated way the procedures adopted by each language for function
calls, registration of C~functions and conversion of data between
the environments.
